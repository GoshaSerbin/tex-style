%\documentclass[a4paper]{article}
%\usepackage{beamerarticle}

\documentclass[9pt, ignoreonframetext,unicode]{beamer}

\usepackage[utf8]{inputenc}
\usepackage[T1]{fontenc}
\usepackage[english,russian]{babel}
\usepackage{amsmath}
\usepackage{amsfonts}
\usepackage{amssymb}
\usepackage{graphicx,pgf}
\usepackage{multimedia}
%\usepackage{hyperref}

%\graphicspath{{./pictures_kurs/}}

%\usetheme{Rochester}  %тема без навигации
%\usetheme{Montpellier} %тема с навигацией в виде дерева
%\usetheme{Berkeley} %тема с оглавлением на полях
%\usetheme{Berlin} %тема с навигацией в виде мини-слайдов

\usetheme{Madrid} %тема с таблицей разделов и подразделов


\useinnertheme{circles}   %внутренняя тема
%\useoutertheme{smoothbars}   %внешняя тема
\usecolortheme{default}     %цветовая схема
%\usecolortheme{beaver}
\usefonttheme{serif}    %шрифты

%\setbeameroption{hide notes}
%
%\setbeamertemplate{bibliography item}{\insertbiblabel}

\usepackage{xcolor}
\usepackage{multirow}
%\usepackage{enumitem}

\title[Исследование эффективности PFEM-2]{Исследование эффективности \\метода конечных элементов с частицами\\ PFEM-2 при решении некоторых задач \\ вычислительной гидродинамики}
\author[Г.\,Э.~Сербин]{Г.\,Э.~Сербин\\Группа ФН2-81Б\\Научный руководитель: А.\,Ю.~Попов}
\institute[МГТУ]{МГТУ им. Н.\,Э.~Баумана}
\date{\today}
\titlegraphic{\includegraphics[width=2cm]{logo.png}}


\begin{document}

\begin{frame}[plain]
\maketitle
\end{frame}



\section{Введение и постановка задачи}


\begin{frame}{Введение}
		\begin{columns}
			%\vspace*{-50mm}
			\begin{column}{0.57\textwidth}
					\begin{block}{Движение вязкой несжимаемой жидкости}	
		\begin{equation*}
	\left\{
	\begin{array}{rcl}
		\rho\biggl(\dfrac{\partial \vec{u}}{\partial t}&+&\left(\vec{u} \cdot \nabla\right) \vec{u}\biggr)=-\nabla p + \mu \Delta \vec{u}+\rho \vec{g},\\
		\nabla \cdot \vec{u} &=& 0,
	\end{array}
	\right.
\end{equation*}
	\end{block}
			\end{column}
			\begin{column}{0.45\textwidth}
\begin{center}
					Граничные условия:
\end{center}					
\vspace*{-5mm}

\begin{align*}
	\left.\vec{u}\right|_{\Gamma_1}&=\vec{u}_0,\quad  \left. p\right|_{\Gamma_2}=p_0;\\
	\left.\dfrac{\partial\vec{u}}{\partial \vec{n}}\right|_{\Gamma_3}&=\overrightarrow{\text{const}},\quad
	\left.\dfrac{\partial p}{\partial \vec{n}}\right|_{\Gamma_4}=\text{const}.
\end{align*}
		\end{column}
		\end{columns}
	\medskip
\begin{block}{Методы частиц}
	\begin{itemize}
		\item Метод сглаженных частиц (SPH, Гингольд и Монаган, 1977);
		\item Метод движущихся частиц (MPS, Кошизука и Ока, 1996);
		\item Метод конечных элементов с частицами (PFEM, Идельсон и Онате, 2004).
	\end{itemize}
\end{block}
\begin{block}{Метод конечных элементов с частицами PFEM-2 (2013)}
\begin{itemize}
\item гибридный эйлерово-лагранжев метод;
\item более крупный шаг по времени, не требует мелких сеток;
\item расчетные области произвольной геометрии.
\end{itemize}
\end{block}
%Представляет интерес оценка эффективности программной реализации, превосходит ли традиционные подходы.

\end{frame}

\section{Метод конечных элементов с частицами PFEM-2}
\begin{frame}{Метод конечных элементов с частицами PFEM-2}
\begin{block}{}
\textbf{Основная идея метода.} Разделение задачи по физическим процессам: конвекция моделируется набором частиц, решение гидродинамической задачи --- методом конечных элементов.
\end{block}
		\begin{block}{Лагранжев шаг}
			Использование частиц для моделирования конвективного слагаемого позволяет решать задачи на грубых сетках и с относительно большим шагом по времени.
			% аппроксимация которого на сетке представляет наибольшую сложность, 
		\end{block}
		\begin{block}{Эйлеров шаг}
			Моделирование влияния вязкости, градиента давления и внешних сил путем решения задачи
методом конечных элементов.
			\end{block}




Необходимо:
\begin{enumerate}
	\item доработать реализацию PFEM-2 для повышения ее быстродействия;
	\item cравнить эффективность реализации PFEM-2 и пакета \texttt{OpenFOAM} на примерах обтекания кругового цилиндра и профиля крыла;	
	\item сравнить эффективность распараллеливания реализаций.
\end{enumerate}

\end{frame}


\begin{frame}{Метод конечных элементов с частицами PFEM-2}
	\begin{enumerate}
		\setlength{\parskip}{0.1pt}
		\item перемещение частиц;
				\[
		\vec{x}_p^{k+1} = \vec{x}_p^k+\tau \vec{u}^m (\vec{x}_p^k),\quad k=\overline{0,\,K-1}.
		\]
		
		\item обновление поля скоростей в узлах сетки:
		\[
		\vec{u}^j=\frac{\sum_{i=1}^k \vec{u}_i N^j(\vec{x}_i)}{\sum_{i=1}^k N^j(\vec{x}_i)}.
		\]
		\item решение методом КЭ задачи
		\begin{equation*}
			\rho\dfrac{ \vec{u}^{m+1}}{\Delta t}=\rho \dfrac{\vec{u}^m}{\Delta t} -\nabla p^{m+1} + \nabla \cdot \hat{\tau}^{m+\theta},\qquad
			\nabla \cdot \vec{u}^{m+1} = 0.
		\end{equation*}
		Метод дробных шагов (расщепление по переменным):
		\begin{enumerate}
			\setlength{\parskip}{7pt}
			\item определить вектор прогноза скорости $\vec{u}^*$ из $\rho \frac{\vec{u}^{*}}{\Delta t}=\rho \frac{\vec{u}^m}{\Delta t}+ \nabla \cdot \hat{\tau}^{m+\theta}$;
			\item найти $p$ из уравнения Пуассона
			$
			\Delta p^{m+1}= \frac{\rho}{\Delta t} \nabla \cdot \vec{u}^{*};
			$
			\item полученное давление подставить в $\rho \frac{\vec{u}^{m+1}}{\Delta t}=\rho \frac{\vec{u}^*}{\Delta t}-\nabla p^{m+1}$.
		\end{enumerate}
		
		\item коррекция скоростей частиц $\delta \vec{v}_p^{m+1}=\sum_{i=1}^4 \left(\vec{u}^{m+1}_i-\vec{u}^m_i\right)N_i(\vec{r}_p).$
		
	\end{enumerate}
	
\end{frame}

%\begin{frame}{Работа с частицами}
%\textbf{Перемещение частиц } вдоль линий тока в новые положения --- на каждом шаге определяется, в какой ячейке находится частица, вычисляется скорость и производится ее перемещение:
%\[
%t_m^k=t_m+k\tau,\qquad \tau=\frac{t_{m+1}-t_m}{K}.
%\]
%
%\[
%\vec{x}_p^{k+1} = \vec{x}_p^k+\tau \vec{v}^m (\vec{x}_p^k),\quad k=0,\,1,\,\ldots,K-1.
%\]
%Движение происходит по фиксированному полю скоростей $\vec{v}^m$.
%\bigskip
%
%\textbf{Обновление поля скоростей в узлах сетки} путем учета вклада от
%всех частиц, находящихся в ячейках, содержащих этот узел:
%\[
%\vec{v}^j=\frac{\sum_{i=1}^k \vec{v}_i\phi^j(\vec{x}_i)}{\sum_{i=1}^k \phi^j(\vec{x}_i)}.
%\]
%
%\end{frame}

%\begin{frame}{Решение гидродинамической задачи}
%\begin{block}{Разностная схема}
%\begin{equation*}
%\left\{
%\begin{array}{rcl}
%\rho\dfrac{ \vec{u}^{m+1}}{\Delta t}&=&\rho \dfrac{\vec{u}^m}{\Delta t} -\nabla p^{m+1} + \nabla \cdot \hat{\tau}^{m+\theta},\\
%\\
%\nabla \cdot \vec{u}^{m+1} &=& 0,
%\end{array}
%\right.
%\end{equation*}
%\end{block}
%где
%\[
%\tau_{ij}=2\mu(\dot{\varepsilon}_{ij}-\delta_{ij}\frac{\dot{\varepsilon}_{kk}}{3}),\,i,\,j=1,\,2,\qquad
%\dot{\varepsilon}_{ij}=\frac{1}{2}\left(\frac{\partial u_i}{\partial x_j}+\frac{\partial u_j}{\partial x_i} \right).
%\]
%\begin{gather}
%\label{refParametrh3}
%\rho \frac{\vec{u}^{*}}{\Delta t}=\rho \frac{\vec{u}^m}{\Delta t}+ \nabla \cdot \hat{\tau}^{m+\theta},\\
%\label{refParametreps}
%\rho \frac{\vec{u}^{m+1}}{\Delta t}=\rho \frac{\vec{u}^*}{\Delta t}-\nabla p^{m+1}.
%\end{gather}
%Метод дробных шагов (расщепление по переменным):
%\begin{enumerate}
%\item Определить вектор прогноза скорости $\vec{u}^*$ из уравнения $\eqref{refParametrh3}$.
%\item Найти $p$ из уравнения Лапласа
%$
%\Delta p^{m+1}= \frac{\rho}{\Delta t} \nabla \cdot \vec{u}^{*}.
%$
%
%\item Полученное давление подставить в $\eqref{refParametreps}$.
%
%\end{enumerate}
%\end{frame}

\section{Тестовые задачи}


\begin{frame}{Тестовая задача: обтекание кругового цилиндра}
%$L_1=2.2$ м, $L_2=0.41$ м. $D=0.1$ м. На входе в канал параболическое распределение скорости, $v_{\max} = 1.5$ м/с в центре канала.
%	 Плотность жидкости $\rho = 1$ кг/м$^3$, коэффициент динамической вязкости $\mu = 0.001$ Па $\cdot$ с.
	
	\vspace*{-9mm}
	\begin{columns}
		%\vspace*{-50mm}
		\begin{column}{0.5\textwidth}
	\begin{figure}[!h]
	\centering
	\includegraphics[scale =0.205]{cylinder 3.png}
	%\caption{Расчетная область}
	\label{picnaca}
\end{figure}
		\end{column}
		\begin{column}{0.5\textwidth}
	Временной шаг $\Delta t = 0.0025$с.\\
	Плотность жидкости $\rho = 1$ кг/м$^3$.\\
	Коэфф-т дин. вязкости $\mu = 0.001$ Па $\cdot$ с.\\
	Число Рейнольдса $\text{Re}=\dfrac{V D}{\nu} =100$.\\
	Сетка из 19 700 ячеек.
		\end{column}
	\end{columns}
\vspace*{-5mm}
\begin{center}
	Результаты до доработки:
\end{center}
Время расчета 1 с. модельного времени \qquad {\color{red} \texttt{OpenFOAM} 656с}, \qquad PFEM-2 717с

\vspace*{-2mm}
				\begin{figure}[!h]
	\centering
	\begin{tabular}{cc}
		\includegraphics[scale =0.2]{Cx.pdf} & \includegraphics[scale =0.2]{Cy.pdf}\\
	\end{tabular}
	\caption{Коэффициенты лобового сопротивления и подъемной силы}
	\label{picCx0}
\end{figure}
\end{frame}


\begin{frame}{Тестовая задача: обтекание крыла NACA-0012}
	%$L_1=2.2$ м, $L_2=0.41$ м. $D=0.1$ м. На входе в канал параболическое распределение скорости, $v_{\max} = 1.5$ м/с в центре канала.
	%Плотность жидкости $\rho = 1$ кг/м$^3$, коэффициент динамической вязкости $\mu = 0.001$ Па $\cdot$ с.	
	\vspace*{-9mm}
	\begin{columns}
		%\vspace*{-50mm}
		\begin{column}{0.55\textwidth}
			\begin{figure}[!h]
				\centering
				\includegraphics[scale =0.205]{naca 4.png}
				\label{picnaca}
			\end{figure}
		\end{column}
		\begin{column}{0.5\textwidth}
			Временной шаг $\Delta t = 0.001$с.\\
			Плотность жидкости $\rho = 1$ кг/м$^3$.\\
			Коэфф-т дин. вязкости $\mu = 0.001$ Па $\cdot$ с.\\
			Число Рейнольдса $\text{Re}=\frac{V D}{\nu} =10 000$.\\
				Сетка из 174 000 ячеек.
		\end{column}
	\end{columns}
\vspace*{-5mm}
%\begin{center}
%	Результаты до доработки:
%\end{center}
%Время расчета 1 с. модельного времени \qquad {\color{red} \texttt{OpenFOAM} 656с}, \qquad PFEM-2 717с
%
%\vspace*{-2mm}
%\begin{figure}[!h]
%	\centering
%	\begin{tabular}{cc}
%		\includegraphics[scale =0.2]{Cx.pdf} & \includegraphics[scale =0.2]{Cy.pdf}\\
%	\end{tabular}
%	\caption{Коэффициенты лобового сопротивления и подъемной силы}
%	\label{picCx0}
%\end{figure}
	

\end{frame}


\begin{frame}{Доработка реализации: сборка глобальных матриц}
	
	\textbf{Ранее реализованный подход к сборке} 

	\begin{enumerate}
	\setlength{\parskip}{0.1pt}
	\item подготовка матриц и векторов правых частей;
		
	\item цикл по времени:
	\begin{enumerate}
		\setlength{\parskip}{7pt}
		\item сборка локальных матриц и векторов правых частей, добавление их в глобальные;
		\item применение граничных условий;
		\item решение системы линейных алгебраических уравнений;	
	\end{enumerate}

\end{enumerate}
	
	\textbf{Доработанный подход к сборке}
	\begin{enumerate}
	\item подготовка матриц и векторов правых частей, параллельная реализация;
	
	\item сборка глобальной матрицы до цикла по времени;
				
	\item цикл по времени:				
		\begin{enumerate}
		\setlength{\parskip}{7pt}
		\item сборка векторов правых частей, учитываются граничные условия;
		\item обмен между вычислительными узлами;
		\item решение системы линейных алгебраических уравнений;
		\item присвоение заданных значений в граничных узлах.
	\end{enumerate}			
		\end{enumerate}
	
\end{frame}

\begin{frame}{Доработка реализации: Г.У. для прогноза скоростей}
		
 Г.У. 1-го рода $\vec{u}|_\Gamma = \vec{u}_0$ для прогнозного поля скоростей
				\begin{align*}
	\rho \frac{\vec{u}^{n+1} - \vec{u}^*}{\Delta t} = -\nabla p \quad \Rightarrow \quad 
	u^*_k|_\Gamma =\left.u^{n+1}_k\right|_\Gamma + \frac{\Delta t}{\rho}\left.\frac{\partial p}{\partial x_k}\right|_\Gamma,\,k=1,\,2.
\end{align*}


\begin{block}{\textbf{Программная реализация}}
	
	\begin{itemize}
		\item[\textcolor{red}{\textbullet}] \textbf{\texttt{FEFieldFunction} (\texttt{deal.II})}.
		
			<<Натягивается>> на все поле давления, используя конечно-элементную аппроксимацию. Не эффективно.
		
		
		\item[\textcolor{green}{\textbullet}] \textbf{Обработка заданных узлов.}
		\begin{enumerate}
		\item Обход только ребер ячеек, лежащих на границе. Значение в гауссовой точке $M$ 
		\begin{equation*}
			\nabla p(\vec{M}) = \sum_{i=1}^4 p_i \nabla N_i(\vec{M}).
		\end{equation*}
		\item Проецирование значений с гауссовых точек на узлы
		\begin{equation*}
			\label{projection}
			\nabla p^j=\frac{\sum_{i=1}^k \nabla p(\vec{M}_i)\,  N^j(\vec{M}_i)}{\sum_{i=1}^k N^j(\vec{M}_i)},
		\end{equation*}
%	 $N^j$ --- функция формы элемента, суммирование по всем гауссовым точкам, вносящим вклад в значение градиента в узле.		
		 (в случае параллельных вычислений необходим обмен данными).
		\end{enumerate}
	
	\end{itemize}
	\end{block}
%\begin{center}
%		\textbf{Программная реализация}
%\end{center}
%\begin{columns}
%	%\vspace*{-50mm}
%	\begin{column}{0.4\textwidth}
%		\begin{block}{\texttt{FEFieldFunction} (\texttt{deal.II})}
%			<<Натягивается>> на все поле давления, используя конечно-элементную аппроксимацию.
%		\end{block}
%	\end{column}
%	\begin{column}{0.5\textwidth}
%		\begin{block}{Обработка заданных узлов}
%Обходятся только ребра ячеек, лежащих на границе. Значение в гауссовой точке $M$ определяется через функции формы по формуле
%\begin{equation*}
%	\nabla p |_\Gamma = \sum_{i=1}^4 p_i \nabla N_i(M).
%\end{equation*}
%Далее выполняется проецирование значений с гауссовых точек на узлы (обусловлено тем что вычисление функций формы и их градиентов в \texttt{deal.II} реализовано в гауссовых точках) по формуле $\eqref{projection}$. 
%\end{block}
%	\end{column}
%\end{columns}





%	\textbf{Программная реализация через \texttt{FEFieldFunction}}
%
%Градиент давления находится с помощью функции \texttt{FEFieldFunction} из библиотеки \texttt{deal.II}, которая <<натягивается>> на все поле давления, используя конечно-элементную аппроксимацию.
%
%	\textbf{Программная реализация через обработку заданных узлов}
%
%Для нахождения градиента давления $\nabla p$ обходятся только ребра ячеек, лежащих на границе. Значение в гауссовой точке $M$ определяется через функции формы по формуле
%\begin{equation*}
%	\nabla p |_\Gamma = \sum_{i=1}^4 p_i \nabla N_i(M).
%\end{equation*}
%Далее выполняется проецирование значений с гауссовых точек на узлы (обусловлено тем что вычисление функций формы и их градиентов в \texttt{deal.II} реализовано в гауссовых точках) по формуле $\eqref{projection}$. 

\end{frame}


\begin{frame}{Сравнение эффективности реализаций: круговой цилиндр}
	\begin{center}
	\begin{tabular}{|c|c|c|c|c|c|c|}\hline	
		& \multirow{2}{*}{$\Delta t$, с}& Время расчета & \multirow{2}{*}{$C_x$} & \multirow{2}{*}{$C_y$} & Ошибка & Ошибка \\ 
		& &   (1 c модел.), с& & & $C_x$, \%    & $C_y$, \% \\ \hline
		{\color{red} \texttt{OpenFOAM}} & 0.0001		& 656 				  & 3.2237 & 0.9654 & 0.195 & 3.46 \\ \hline
		PFEM-2 & 0.0025			& 717   	  & 3.2513 & 0.9496 & 0.659 & 5.04\\ \hline
		{\color{blue} PFEM-2 upd. } & 0.002	& 449   			  & 3.2337 & 0.9665 & 0.115 & 3.35\\ \hline
	\end{tabular}
\end{center}
\begin{center}
	Таблица 1. Результаты работы программ
\end{center}

\vspace*{-2mm}
\begin{figure}[!h]
	\centering
	\begin{tabular}{cc}
		\includegraphics[scale =0.2]{Cx500.pdf} & 	\includegraphics[scale =0.2]{Cy500.pdf}\\
		 $C_x$ & $C_y$\\
	\end{tabular}
	\caption{Коэффициенты лобового сопротивления и подъемной силы}
	\label{picCx0}
\end{figure}

\end{frame}

\begin{frame}{Сравнение эффективности реализаций: крыло NACA-0012}
	
	\begin{center}
	\begin{tabular}{|c|c|c|c|c|c|c|}\hline	
		& \multirow{2}{*}{$\Delta t$, с}&Время расчета& \multirow{2}{*}{$\overline{C}_x$} & \multirow{2}{*}{$\overline{C}_y$} & Ошибка & Ошибка \\ 
		& &(0.05c модел.), с& & & $\overline{C}_x$, \%    & $\overline{C}_y$, \% \\ \hline
		{\color{red}OpenFOAM}& 0.00005		& 445 				  & 0.0444 & 0.127 & 7.5 & 9.4 \\ \hline
		{\color{blue}PFEM-2 upd.}& 0.001	& 575   				  & 0.0437 & 0.105 & 9.0 & 25.0\\ \hline
	\end{tabular}
\end{center}
\begin{center}
	Таблица 2. Результаты работы программ
\end{center}

	\vspace*{-2mm}
	\begin{figure}[!h]
		\centering
		\begin{tabular}{cc}
		\includegraphics[scale =0.25]{CxNACA500.pdf} & \includegraphics[scale =0.25]{CyNACA500.pdf} \\
		\end{tabular}
		\caption{Коэффициенты лобового сопротивления и подъемной силы}
		\label{picCx0}
	\end{figure}
	
\end{frame}

\begin{frame}{Сравнение эффективности параллелизации}
	Кластер кафедры <<Прикладная математика>> МГТУ им. Н.Э. Баумана: \\
	6 вычислительных узлов, каждый состоит из 18-ядерного процессора \\ Intel Core i9-10980XE и 128 ГБ ОЗУ.
	\begin{figure}[!h]
	\centering
	\begin{tabular}{cc}
		\includegraphics[scale =0.25]{Para1.pdf} & 	\includegraphics[scale =0.25]{Para2.pdf} \\
		сетка из 174 000 ячеек &  сетка из 562 500 ячеек 
	\end{tabular}
	\caption{Ускорение программ: \textbf{Максимальное}, {\color{red}OpenFOAM}, {\color{blue}PFEM-2 upd.}}
	\label{picAcc}
\end{figure}

Время работы на одном узле для сетки из 174 000 ячеек $\approx$ 11 минут, для сетки из 562 500 ячеек $\approx$ 36 минут.
\end{frame}



\section{Заключение}
\begin{frame}{Заключение}
	\begin{itemize}
\item Рассмотрен общий подход к решению задач гидродинамики методом конечных элементов с частицами PFEM-2, реализация алгоритма доработана для повышения ее быстродействия;
\item Выполнены вычислительные эксперименты, сравнены эффективность реализации PFEM-2 (с использованием \texttt{deal.II}) и пакета \texttt{OpenFOAM} для разных параметров на примерах обтекания кругового цилиндра и профиля крыла;

\item Рассмотрена эффективность распараллеливания реализаций на кластерной системе;

\item Метод PFEM-2 показал свою эффективность и конкурентоспособность в сравнении с традиционными подходами.
	\end{itemize}
\end{frame}

%\section{Список использованных источников}
%\begin{frame}{Список использованных источников}
%\begin{thebibliography}{4}
%	
%\bibitem{XXItom}
%Зарубин В.С. Математическое моделирование в технике: Учеб. для вузов. М.: Изд-во МГТУ им. Н.Э. Баумана, 2003. 495 с.
%
%\bibitem{Moiseev}
%Моисеев Н.Н. Математика ставит эксперимент. М.:Наука, 1979. 222 с.
%
%\bibitem{PFEM2013}
%Idelsohn S.R., Nigro N.M., Gimenez J.M., Rossi R., Marti J.M. A
%fast and accurate method to solve the incompressible Navier-Stokes
%equations // Engineering Computations. 2013. V.30, No. 2. Pp.197-222. doi:
%10.1108/02644401311304854
%
%\bibitem{PFEM2014}
%Idelsohn S.R., Nigro N.M., Gimenez J.M., Rossi R., Marti J.M.
%Evaluating the perfomance of particle finite element method in parallel architectures //Springer Publishing. 2014. Pp.103-116. doi:10.1007/s40571-014-0009-4
%%\bibitem{flow}
%%Schäfer M., Turek S., Durst F., Krause E., Rannacher R. Benchmark computations of laminar flow around a cylinder // Flow Simulation with High-Performance Computers II. Notes on Numerical Fluid Mechanics (NNFM), 1996. V. 48. P. 547--566.
%
%\bibitem{mss}
%Зарубин В.С., Кувыркин Г.Н. Математические модели механики и электродинамики сплошной среды. М.: Изд-во МГТУ им. Н.Э. Баумана, 2008.
%512 с.
%
%\bibitem{dealii}
%The deal.II Finite Element Library. Home page. [Электронный ресурс]. URL:
%http://www.dealii.org/ (дата обращения: 26.06.2021).
%\end{thebibliography}
%\end{frame}


\end{document} 
